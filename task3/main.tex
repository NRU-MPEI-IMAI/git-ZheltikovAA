\documentclass[fleqn]{article}
\usepackage[utf8]{inputenc}
 \usepackage[russian]{babel}
 \usepackage[pdf]{graphviz}
 \usepackage{mathtools}
\title{Отчет по домашней работе номер 1}
\author{Александр Желтиков}
\usepackage[pdf]{graphviz}
\usepackage{xpatch}
\makeatletter
\newcommand*{\addFileDependency}[1]{% argument=file name and extension
  \typeout{(#1)}
  \@addtofilelist{#1}
  \IfFileExists{#1}{}{\typeout{No file #1.}}
}
\makeatother
\xpretocmd{\digraph}{\addFileDependency{#2.dot}}{}{}
\begin{document}

\chapter{\textbf{\large{Задание номер 3.}}} \\
\textbf{Построить минимальный ДКА по регулярному выражению.}
\\

Ответом на данное задание является минимальный ДКА, который допускает тот же язык, что описывается регулярным выражением.

\begin{equation}
1. \quad (ab + aba)^*a  \nonumber
\end{equation}

Построим НКА:

\digraph[scale=0.7]{1}{
        rankdir=LR;
        node [shape = none] start [label = < >];
        node [shape = circle] q0;
        node [shape = circle] q1;
        node [shape = circle] q2;
        node [shape = circle] q3;
        node [shape = circle] q4;
        node [shape = circle] q5;
        node [shape = circle] q6;
        node [shape = doublecircle] q7;
        start -> q0;
        q0 -> q1 [label = "lambda"];
        q1 -> q0 [label = "lambda"]; 
        q0 -> q2 [label = "lambda"];
        q2 -> q3 [label = "a"]; 
        q3 -> q4 [label = "b"];
        q4 -> q1 [label = "a"];
        q0 -> q5 [label = "lambda"];
        q5 -> q6 [label = "a"];
        q6 -> q1 [label = "b"];
        q1 -> q7 [label = "a"];
        
    }

Далее мы избавляемся от lambda переходов (Да, я знаю что в Tex есть символ \lambda)

\digraph[scale=0.7]{2}{
        rankdir=LR;
        node [shape = none] start [label = < >];
        node [shape = circle] q0;
        node [shape = circle] q1;
        node [shape = circle] q2;
        node [shape = circle] q3;
        node [shape = doublecircle] q4;
        start -> q0;
        q0 -> q1 [label = "a"]; 
        q1 -> q0 [label = "b"];
    `   q0 -> q2 [label = "a"];
        q2 -> q3 [label = "b"];
        q3 -> q0 [label = "a"];
        q0 -> q4 [label = "a"];
    }
    
    Далее пересчитаем узлы: 
    
\begin{tabular}{ | l | l | l | }
\hline
- & a & b \\ \hline
q0 & q1q2q4  & - \\
q1q2q4 & - & q0q3 \\
q0q3 & q0q1q2q4 & - \\
q0q1q2q4 & q1q2q4 & q0q3 \\
\hline
\end{tabular}
\\

\{Я ПРИЛОЖУ ОТДЕЛЬНО ФОТОГРАФИЮ МИНИМИЗИРОВАННОГО ДКА, ПОТОМУ ЧТО У МЕНЯ ПРОСТО УЖЕ ГОРИТ ИЗ-ЗА ТОГО, ЧТО Я ТРАЧУ 4 ЧАСА НА ТО ЧТОБЫ НАЧЕРТИТЬ ДКА...\}

\\

\digraph[scale=0.7]{abc2}{
        rankdir=LR;
        node [shape = circle] q0;
        node [shape = circle] q1;
        node [shape = doublecircle] q2;
        node [shape = doublecircle] q3;
        q0 -> q1 [label = "a"]; 
        q1 -> q2 [label = "b"];  `    
        q2 -> q3 [label = "a"];
        q3 -> q2 [label = "b"];
        q3 -> q1 [label = "a"];
    }
    
 \begin{equation}
2. \quad a(a(ab)^*b)^*(ab)^*  \nonumber
\end{equation}   

Построим НКА:
\\
З.Ы. 3\_2 НКА!
\\
Далее построим эквивалентный ДКА

\begin{tabular}{ | l | l | l | }
\hline
- & a & b \\ \hline
q0 & q1  & - \\
q1 & q2q4 & - \\
q2q4 & q3 & - \\
q3 & - & q2 \\
q2 & q3 & q1\\
\hline
\end{tabular}
Далее мы минимизируем полученынй автомат.

Ответ на картинках.
\end{tabular}

\end{document}
