\documentclass[fleqn]{article}
\usepackage[utf8]{inputenc}
 \usepackage[russian]{babel}
 \usepackage[pdf]{graphviz}
 \usepackage{mathtools}
\title{Отчет по домашней работе номер 1}
\author{Александр Желтиков}
\usepackage[pdf]{graphviz}
\usepackage{xpatch}
\makeatletter
\newcommand*{\addFileDependency}[1]{% argument=file name and extension
  \typeout{(#1)}
  \@addtofilelist{#1}
  \IfFileExists{#1}{}{\typeout{No file #1.}}
}
\makeatother
\xpretocmd{\digraph}{\addFileDependency{#2.dot}}{}{}
\begin{document}

\chapter{\textbf{\large{Задание номер 4.}}} \\
\textbf{Определить является ли язык регулярным или нет}
\\
Ответом на данное задание является конечный автомат, если язык регулярен, либо доказательство нерегулярности языка при помощи леммы о
разрастании.

Запишем лемму о разрастании, что бы в дальнейшем просто использовать её упоминание.

    \textbf{Пусть L -- Регулярный язык над алфавитом $\Sigma$ тогда существует такое n, что для любого слова w $\in$ L длины не меньше n найдутся слова x,y,z $\in$ $\Sigma$*, для которых верно: xyz = w, y $\neq$ $\lambda$ , |xy| $\leq$ n , и $\forall$ k $\geq$ 0, xy^kz $\in$ L}



\begin{equation}
1. \quad L = \{(aab)^nb(aba)^m | \; n \geq 0, m \geq 0 \}    \nonumber
\end{equation}

Проверим язык на регулярность:

Фиксируем значение n;\\
w = (aab)^nb(aba)^n   \qquad  |w| \geq n

x = (aab)^p  \qquad\qquad \ |xy| $\leq$ n  ; \; l > 0 ;

y = (aab)^l

z = (aab)^{n-p-l}b(aba)^m
\\

При любом значении k для xy^kz \in L

Построим теперь ДКА для данного языка.

\digraph[scale=0.5]{3}{
        rankdir=LR;
        node [shape = none] start [label = < >];
        node [shape = circle] q0;
        node [shape = circle] q1;
        node [shape = circle] q2;
        node [shape = circle] q3;
        node [shape = circle] q4;
        node [shape = circle] q5;
        node [shape = circle] q6;
        node [shape = doublecircle] q7;
        start -> q0;
        q0 -> q1 [label = "a"];
        q1 -> q2 [label = "a"];
        q2 -> q3 [label = "b"];
        q3 -> q0 [label = "lambda" ];
        q0 -> q3 [label = "lambda" ];
        q3 -> q4 [label = "b"];
        q4 -> q5 [label = "a"];
        q5 -> q6 [label = "b"];
        q6 -> q7 [label = "a"];
        q7 -> q4 [label = "lambda"];
        q4 -> q7 [label = "lambda"];
    }

\begin{equation}
2. \quad L = \{ uaav \; |u \in \{a,b\}^* , v \in a,b^* , |u|_b \geq |v|_a \}    \nonumber
\end{equation}

Проверим язык на регулярность с помощью отрицания леммы о разрастании:

Фиксируем значение n;\\

w = b^naaa^n \qquad |w| \geq n

x = b^p  \qquad\qquad \ |xy| $\leq$ n  ;\; l > 0 ;

y = b^l

z = b^{n-p-l}aaa^n

При накачке "y" если мы возьмем(положим), что k = 0, то к нас выйдет что xy^kz = b^pb^{n-p-l}aaa^n = b^{n-l}aaa^n \notin L 

Отрицание леммы выполнено => язык данный нам по условию не является регулярным.

\begin{equation}
3. \quad L = \{ a^m w \; | \; w \in \{a,b\}^* , 1 $\leq$ |w|_b $\leq$ m\}    \nonumber
\end{equation}

Проверим язык на регулярность используя отрицание леммы о разрастании.

Фиксируем значение n.

w = a^n b^n  \qquad |w| \geq n

x = a^p  \qquad\qquad \ |xy| $\leq$ n  ; \; p \geq 0 ;\; l > 0 ;

y = a^l

z = a^{n-p-l}b^n

Если мы возьмём k = 0, то степень у символа "a" будет меньше чем у символа "b":

xy^0z  = xz = a^{n-l}b^n $\notin$ L, а из этого следует, что язык не является регулярным.

\begin{equation}
4. \quad L = \{ a^kb^ma^n \; | \; k = n \vee m > 0\}    \nonumber
\end{equation}

Проверим язык на регулярность используя отрицание леммы о разрастании.

Фиксируем значение n.

w = a^n b a^n \qquad |w| \geq n

x = a^p  $\qquad\qquad$ \ |xy| $\leq$ n  => p + l $\leq$ n ;\; l > 0 ;

y = a^l

z = a^{n-p-l}ba^n

При k = 0 или 1 у нас у первого символа "a" будет константа которая меньше n и она может принадлежать языку.

Если k = 2, то xy^2z = a^pa^{2l}a^{n-p-l}ba^n = a^{n+l}ba^n $\notin$ L, что и требовалось доказать, язык не является регулярным.

\begin{equation}
5. \quad L = \{ucv \; | \; u \in \{a,b\}^*, \; v \in \{a,b\}^*, u \neq v^R \}    \nonumber
\end{equation}

Проверим язык на регулярность используя отрицание леммы о разрастании.

Фиксируем значение n.

w = (ab)^n c (ab)^n \qquad |w| \geq n

x = $\alpha$_1$\alpha$_2...$\alpha$_p  $\qquad\qquad$ \ |xy| $\leq$ n  =>  p+l $\leq$ n ; \; l $\neq$ 0;

y = $\alpha$_{p+1}$\alpha$_{p+2}...$\alpha$_{P+l}

z = $\alpha$_{p+l+1}$\alpha$_{p+l+2}...$\alpha$_{2n} c (ab)^n

Если k = 2, то xy^2z = ($\alpha$_1$\alpha$_2...$\alpha$_p)($\alpha$_{p+1}$\alpha$_{p+2}...$\alpha$_{P+l})^2($\alpha$_{p+l+1}$\alpha$_{p+l+2}...$\alpha$_{2n} c (ab)^n) $\notin$ L, что и требовалось доказать, язык не является регулярным.

\end{document}
