\documentclass[fleqn]{article}
\usepackage[utf8]{inputenc}
 \usepackage[russian]{babel}
 \usepackage[pdf]{graphviz}
 \usepackage{mathtools}
\title{Отчет по домашней работе номер 1}
\author{Александр Желтиков}
\usepackage[pdf]{graphviz}
\usepackage{xpatch}
\makeatletter
\newcommand*{\addFileDependency}[1]{% argument=file name and extension
  \typeout{(#1)}
  \@addtofilelist{#1}
  \IfFileExists{#1}{}{\typeout{No file #1.}}
}
\makeatother
\xpretocmd{\digraph}{\addFileDependency{#2.dot}}{}{}

\begin{document}

\chapter{\textbf{\large{Задание номер 2.}}} \\
\textbf{Построить конечный автомат, используя прямое произведение}
\\

Ответом на данное задание является конечный автомат, распознающий описанный язык. Требуется, чтобы он был построен при помощи прямого произведения ДКА и его свойств.

\begin{equation}
1. \quad L_1 = \{ w \in \{a,b\} \; | \; | |w|_a \geq 2 \wedge |w|_b \geq 2 \}    \nonumber
\end{equation}

Рассмотрим отдельно автоматы: 

L = \{ w $\in$ \{a,b\} \; | \; | |w|_a $\geq$ 2 \}

Построим ДКА для данного языка:

\digraph[scale=0.7]{5}{
        rankdir=LR;
        node [shape = none] start [label = < >];
        node [shape = circle] q0;
        node [shape = circle] q1;
        node [shape = doublecircle] q2;
        start -> q0;
        q0 -> q1 [label = a];
        q1 -> q2 [label = a];
        q0 -> q0 [label = "b"];
        q1 -> q1 [label = "b"];
        q2 -> q2 [label = "a,b"];
        
    }
    
L = \{ w $\in$ \{a,b\} \; | \; | |w|_b $\geq$ 2 \}

Построим ДКА для данного языка:

\digraph[scale=0.7]{6}{
        rankdir=LR;
        node [shape = none] start [label = < >];
        node [shape = circle] q0;
        node [shape = circle] q1;
        node [shape = doublecircle] q2;
        start -> q0;
        q0 -> q1 [label = b];
        q1 -> q2 [label = b];
        q0 -> q0 [label = "a"];
        q1 -> q1 [label = "a"];
        q2 -> q2 [label = "a,b"];
        
    }
 
 A_1 =  \{ w $\in$ \{a,b\} \; | \; | |w|_A $\geq$ 2 \} $\Sigma$_A = \{a,b\}, Q_A = {q0, q1, q2} , s_A = \{ q0 \} , T_A = \{ q2 \}.
 
 B_1 =  \{ w $\in$ \{a,b\} \; | \; | |w|_B $\geq$ 2 \} $\Sigma$_B = \{a,b\}, Q_B = {q0, q1, q2} , s_B = \{ q0 \} , T_B = \{ q2 \}.
 
 Запишем прямое произведение А_1 В_1.
 
 L_1 = A_1 * B_1
 
 $\sigma$ = $\Sigma$_A $\cup$ $\Sigma$_B = \{ a, b\}
 
 Q = Q_A * Q_B = \{q0q0,q0q1,q0q2,q1q0,q1q1,q1q2,q2q0,q2q1,q2q2 \}
 
 s = <s_A, s_B> = \{q0q0\}
 
 T = <T_A , T_B> = \{q2q2\}
 \\
 
 Построим автомат для данного произведения.
 
 \begin{tabular}{ | l | l | l | }
\hline
L1 & a & b \\ \hline
q0q0 & q1q0  & q0q1 \\
q0q1 & q1q1  & q0q2 \\
q0q2 & q1q2  & q0q2 \\
q1q0 & q2q0  & q1q1 \\
q1q1 & q2q1  & q1q2 \\
q1q2 & q2q2  & q1q2 \\
q2q0 & q2q0  & q2q1 \\
q2q1 & q2q1  & q2q2 \\
q2q2 & q2q2  & q2q2 \\
\hline
\end{tabular}

\digraph[scale=0.7]{15}{
        rankdir=LR;
        node [shape = none] start [label = < >];
        node [shape = circle] q0q0;
        node [shape = circle] q0q1;
        node [shape = circle] q0q2;
        node [shape = circle] q1q0;
        node [shape = circle] q1q1;
        node [shape = circle] q1q2;
        node [shape = circle] q2q0;
        node [shape = circle] q2q1;
        node [shape = doublecircle] q2q2;
        start -> q0q0;
        q0q0 -> q1q0 [label = a]; q0q0 -> q0q1 [label = b];
        q0q1 -> q1q1 [label = a]; q0q1 -> q0q2 [label = b];
        q0q2 -> q1q2 [label = a]; q0q2 -> q0q2 [label = b];
        q1q0 -> q2q0 [label = a]; q1q0 -> q1q1 [label = b];
        q1q1 -> q2q1 [label = a]; q1q1 -> q1q2 [label = b];
        q1q2 -> q2q2 [label = a]; q1q2 -> q1q2 [label = b];
        q2q0 -> q2q0 [label = a]; q2q0 -> q2q1 [label = b];
        q2q1 -> q2q1 [label = a]; q2q1 -> q2q2 [label = b];
        q2q2 -> q2q2 [label = "a , b"]; 
        
    }

\begin{equation}
2. \quad L_2 = \{ w \in \{a,b\}^* \; | \; | |w|\geq 3 \wedge |w| \ \textit{нечётное} \}    \nonumber
\end{equation}

Рассмотрим отдельно атоматы.

L = \{ w \in \{a,b\}^* \; | \; | |w|\geq 3 \}

Построим ДКА для данного языка:

\digraph[scale=0.7]{8}{
        rankdir=LR;
        node [shape = none] start [label = < >];
        node [shape = circle] q0;
        node [shape = circle] q1;
        node [shape = circle] q2;
        node [shape = doublecircle] q3;
        start -> q0;
        q0 -> q1 [label = "a,b"];
        q1 -> q2 [label = "a,b"];
        q2 -> q3 [label = "a,b"];
        q3 -> q3 [label = "a,b"];
    }

L = \{ w \in \{a,b\}^* \; | \; | |w| \ \textit{нечётное} \}

Построим ДКА для данного языка:

\digraph[scale=0.7]{9}{
        rankdir=LR;
        node [shape = none] start [label = < >];
        node [shape = circle] q0;
        node [shape = doublecircle] q1;
        start -> q0;
        q0 -> q1 [label = "a,b"];
        q1 -> q0 [label = "a,b"];
            }
            

A_1 =  \{ w \in \{a,b\}^* \; | \; | |w|\geq 3 \} $\Sigma$_A = \{a,b\}, Q_A = \{q0, q1, q2, q3\} , s_A = \{ q0 \} , T_A = \{ q3 \}.
 
 B_1 =  \{ w \in \{a,b\}^* \; | \; | |w| \ \textit{нечётное} \} $\Sigma$_B = \{a,b\}, Q_B = \{q0, q1\} , s_B = \{ q0 \} , T_B = \{ q1 \}.
 
 Запишем прямое произведение А_1 В_1.
 
 L_1 = A_1 * B_1
 
 $\sigma$ = $\Sigma$_A $\cup$ $\Sigma$_B = \{ a, b\}
 
 Q = Q_A * Q_B = \{q0q0,q0q1,q1q0,q1q1,q2q0,q2q1,q3q0,q3q0 \}
 
 s = <s_A, s_B> = \{q0q0\}
 
 T = <T_A , T_B> = \{q3q1\}
 \\
 
 Построим автомат для данного произведения.
 
 \begin{tabular}{ | l | l | l | }
\hline
L1 & a & b \\ \hline
q0q0 & q1q1  & q1q1 \\
q0q1 & q1q0  & q1q0 \\
q1q0 & q2q1  & q2q1 \\
q1q1 & q2q0  & q2q0 \\
q2q0 & q3q1  & q3q1 \\
q2q1 & q3q0  & q3q0 \\
q3q0 & q3q1  & q3q1 \\
q3q1 & q3q0  & q3q0 \\
\hline
\end{tabular}

\digraph[scale=0.7]{7}{
        rankdir=LR;
        node [shape = none] start [label = < >];
        node [shape = circle] q0q0;
        node [shape = circle] q0q1;
        node [shape = circle] q1q0;
        node [shape = circle] q1q1;
        node [shape = circle] q2q0;
        node [shape = circle] q2q1;
        node [shape = circle] q3q0;
        node [shape = doublecircle] q3q1;
        start -> q0q0;
        q0q0 -> q1q1 [label = "a,b"];
        q0q1 -> q1q0 [label = "a,b"]; 
        q1q0 -> q2q1 [label = "a,b"]; 
        q1q1 -> q2q0 [label = "a,b"]; 
        q2q0 -> q3q1 [label = "a,b"]; 
        q2q1 -> q3q0 [label = "a,b"];
        q3q0 -> q3q1 [label = "a,b"]; 
        q3q1 -> q3q0 [label = "a,b"]; 

    }

\digraph[scale=0.7]{14}{
        rankdir=LR;
        node [shape = none] start [label = < >];
        node [shape = circle] q0q0;
        node [shape = circle] q1q1;
        node [shape = circle] q2q0;
        node [shape = circle] q3q0;
        node [shape = doublecircle] q3q1;
        start -> q0q0;
        q0q0 -> q1q1 [label = "a,b"];
        q1q1 -> q2q0 [label = "a,b"]; 
        q2q0 -> q3q1 [label = "a,b"]; 
        q3q0 -> q3q1 [label = "a,b"]; 
        q3q1 -> q3q0 [label = "a,b"]; 

    }

\begin{equation}
3. \quad L_3 = \{ w \in \{a,b\}^* \; | \; | |w|_a \textit{чётно} \wedge |w|_b \ \textit{кратно трём} \}    \nonumber
\end{equation}

Рассмотрим отдельно атоматы.

L = \{  w \in \{a,b\}^* \; | \; | |w|_a \textit{чётно} \}

Построим ДКА для данного языка:

\digraph[scale=0.7]{10}{
        rankdir=LR;
        node [shape = none] start [label = < >];
        node [shape = circle] q1;
        node [shape = doublecircle] q0;
        start -> q0;
        q0 -> q1 [label = "a"];
        q1 -> q0 [label = "a"];
        q0 -> q0 [label = b];
        q1 -> q1 [label = b];
            }

L = \{ w \in \{a,b\}^* \; | \; | |w|_b \ \textit{кратно трём} \}

Построим ДКА для данного языка:

\digraph[scale=0.7]{11}{
        rankdir=LR;
        node [shape = none] start [label = < >];
        node [shape = circle] q2;
        node [shape = circle] q1;
        node [shape = doublecircle] q0;
        start -> q0;
        q0 -> q1 [label = "b"];
        q1 -> q2 [label = "b"];
        q2 -> q0 [label = "b"];
        q0 -> q0 [label = a];
        q1 -> q1 [label = a];
        q2 -> q2 [label = a];
    }

A_1 =  \{ w \in \{a,b\}^* \; | \; | |w|_a \textit{чётно} \} $\Sigma$_A = \{a,b\}, Q_A = \{q0, q1\} , s_A = \{ q0 \} , T_A = \{ q0 \}.
 
 B_1 =  \{ w \in \{a,b\}^* \; | \; | |w|_b \ \textit{кратно трём} \} $\Sigma$_B = \{a,b\}, Q_B = \{q0, q1, q2\} , s_B = \{ q0 \} , T_B = \{ q0 \}.
 
 Запишем прямое произведение А_1 В_1.
 
 L_1 = A_1 * B_1
 
 $\Sigma$ = $\Sigma$_A $\cup$ $\Sigma$_B = \{ a, b\}
 
 Q = Q_A * Q_B = \{q0q0,q0q1,q0q2,q1q0,q1q1,q1q2\}
 
 s = <s_A, s_B> = \{q0q0\}
 
 T = <T_A , T_B> = \{q0q0\}
 \\
 
 Построим автомат для данного произведения.
 
 \begin{tabular}{ | l | l | l | }
\hline
L1 & a & b \\ \hline
q0q0 & q1q0  & q0q1 \\
q0q1 & q1q1  & q0q2 \\
q0q2 & q1q2  & q0q0 \\
q1q0 & q0q0  & q1q1 \\
q1q1 & q0q1  & q1q2 \\
q1q2 & q0q2  & q1q0 \\
\hline
\end{tabular}

\digraph[scale=0.5]{20}{
        rankdir=LR;
        node [shape = none] start [label = < >];
        node [shape = doublecircle] q0q0;
        node [shape = circle] q0q1;
        node [shape = circle] q0q2;
        node [shape = circle] q1q0;
        node [shape = circle] q1q1;
        node [shape = circle] q1q2;
        start -> q0q0;
        q0q0 -> q1q0 [label = "a"]; q0q0 -> q0q1 [label = "b"];
        q0q1 -> q1q1 [label = "a"]; q0q1 -> q0q2 [label = "b"];
        q0q2 -> q1q2 [label = "a"]; q0q2 -> q0q0 [label = "b"];
        q1q0 -> q0q0 [label = "a"]; q1q0 -> q1q1 [label = "b"];
        q1q1 -> q0q1 [label = "a"]; q1q1 -> q1q2 [label = "b"];
        q1q2 -> q0q2 [label = "a"]; q1q2 -> q1q0 [label = "b"];

    }

\begin{equation}
4. \quad L_4 = \bar L_3    \nonumber
\end{equation}

 L_4 = A_1 * B_1
 
 $\Sigma$ = $\Sigma$_A $\cup$ $\Sigma$_B = \{ a, b\}
 
 Q = Q_A * Q_B = \{q0q0,q0q1,q0q2,q1q0,q1q1,q1q2\}
 
 s = <s_A, s_B> = \{q0q0\}
 
 T = <T_A , T_B> = \{q0q1,q0q2,q1q0,q1q1,q1q2\}
 \\

 \begin{tabular}{ | l | l | l | }
\hline
L1 & a & b \\ \hline
q0q0 & q1q0  & q0q1 \\
q0q1 & q1q1  & q0q2 \\
q0q2 & q1q2  & q0q0 \\
q1q0 & q0q0  & q1q1 \\
q1q1 & q0q1  & q1q2 \\
q1q2 & q0q2  & q1q0 \\
\hline
\end{tabular}

\digraph[scale=0.5]{0}{
        rankdir=LR;
        node [shape = none] start [label = < >];
        node [shape = circle] q0q0;
        node [shape = doublecircle] q0q1;
        node [shape = doublecircle] q0q2;
        node [shape = doublecircle] q1q0;
        node [shape = doublecircle] q1q1;
        node [shape = doublecircle] q1q2;
        start -> q0q0;
        q0q0 -> q1q0 [label = "a"]; q0q0 -> q0q1 [label = "b"];
        q0q1 -> q1q1 [label = "a"]; q0q1 -> q0q2 [label = "b"];
        q0q2 -> q1q2 [label = "a"]; q0q2 -> q0q0 [label = "b"];
        q1q0 -> q0q0 [label = "a"]; q1q0 -> q1q1 [label = "b"];
        q1q1 -> q0q1 [label = "a"]; q1q1 -> q1q2 [label = "b"];
        q1q2 -> q0q2 [label = "a"]; q1q2 -> q1q0 [label = "b"];
    }
    
    
\begin{equation}
5. \quad L_5 = L_2 \backslash L_3 \nonumber
\end{equation}

L_5 = L_2 * L_4

ДКА для L_2

\digraph[scale=0.7]{14}{
        rankdir=LR;
        node [shape = none] start [label = < >];
        node [shape = circle] q0;
        node [shape = circle] q1;
        node [shape = circle] q2;
        node [shape = circle] q4;
        node [shape = doublecircle] q3;
        start -> q0;
        q0 -> q1 [label = "a,b"];
        q1 -> q2 [label = "a,b"]; 
        q2 -> q3 [label = "a,b"]; 
        q4 -> q3 [label = "a,b"]; 
        q3 -> q4 [label = "a,b"]; 

    }

ДКА для L_4

\digraph[scale=0.5]{21}{
        rankdir=LR;
        node [shape = none] start [label = < >];
        node [shape = circle] q0;
        node [shape = doublecircle] q1;
        node [shape = doublecircle] q2;
        node [shape = doublecircle] q3;
        node [shape = doublecircle] q4;
        node [shape = doublecircle] q5;
        start -> q0;
        q0 -> q3 [label = "a"]; q0 -> q1 [label = "b"];
        q1 -> q4 [label = "a"]; q1 -> q2 [label = "b"];
        q2 -> q5 [label = "a"]; q2 -> q0 [label = "b"];
        q3 -> q0 [label = "a"]; q3 -> q4 [label = "b"];
        q4 -> q1 [label = "a"]; q4 -> q5 [label = "b"];
        q5 -> q2 [label = "a"]; q5 -> q3 [label = "b"];
    }
 
 
 L_5 = L_2 * L_4
 
 $\sigma$ = $\Sigma$_A $\cup$ $\Sigma$_B = \{ a, b\}
 
 Q = Q_A * Q_B = \{\\
 q0q0,q0q1,q0q2,q0q3,q0q4,q0q5,\\
                q1q0,q1q1,q1q2,q1q3,q1q4,q1q5,\\
                q2q0,q2q1,q2q2,q2q3,q2q4,q2q5,\\
                q3q0,q3q1,q3q2,q3q3,q3q4,q3q5,\\
                q4q0,q4q1,q4q2,q4q3,q4q4,q4q5,\\
                    \}
 
 s = <s_A, s_B> = \{q0q0\}
 
 T = <T_A , T_B> = \{q3q1,q3q2,q3q3,q3q4,q3q5\}
 \\
 
 Построим автомат для данного произведения.
 
 \begin{tabular}{ | l | l | l | }
\hline
L1 & a & b \\ \hline
q0q0 &  q1q3 & q1q1 \\
q0q1 &  q1q0 & q1q4 \\
q0q2 &  q1q5 & q1q0 \\
q0q3 &  q1q2 & q1q3 \\
q0q4 &  q1q1 & q1q5 \\
q0q5 &  q1q4 & q1q2 \\
q1q0 &  q1q3 & q1q1 \\
q1q1 &  q1q0 & q1q4 \\
q1q2 &  q1q5 & q1q0 \\
q1q3 &  q1q2 & q1q3 \\
q1q4 &  q1q1 & q1q5 \\
q1q5 &  q1q4 & q1q2 \\
q2q0 &  q1q3 & q1q1 \\
q2q1 &  q1q0 & q1q4 \\
q2q2 &  q1q5 & q1q0 \\
q2q3 &  q1q2 & q1q3 \\
q2q4 &  q1q1 & q1q5 \\
q2q5 &  q1q4 & q1q2 \\
q3q0 &  q1q3 & q1q1 \\
q3q1 &  q1q0 & q1q4 \\
q3q2 &  q1q5 & q1q0 \\
q3q3 &  q1q2 & q1q3 \\
q3q4 &  q1q1 & q1q5 \\
q3q5 &  q1q4 & q1q2 \\
q4q0 &  q1q3 & q1q1 \\
q4q1 &  q1q0 & q1q4 \\
q4q2 &  q1q5 & q1q0 \\
q4q3 &  q1q2 & q1q3 \\
q4q4 &  q1q1 & q1q5 \\
q4q5 &  q1q4 & q1q2 \\

\hline
\end{tabular}

\newpage
    
\digraph[scale=0.5]{25}{
rankdir=LR;
        node [shape = none] start [label = < >];
        node [shape = circle] q0q0;
        node [shape = circle] q0q1;
        node [shape = circle] q0q2;
        node [shape = circle] q0q3;
        node [shape = circle] q0q4;
        node [shape = circle] q0q5;
        node [shape = circle] q1q0;
        node [shape = circle] q1q1; 
        node [shape = circle] q1q2; 
        node [shape = circle] q1q3; 
        node [shape = circle] q1q4;
        node [shape = circle] q1q5;
        node [shape = circle] q2q0;
        node [shape = circle] q2q1;
        node [shape = circle] q2q2;
        node [shape = circle] q2q3;
        node [shape = circle] q2q4;
        node [shape = circle] q2q5;
        node [shape = circle] q3q0;
        node [shape = doublecircle] q3q1;
        node [shape = doublecircle] q3q2;
        node [shape = doublecircle] q3q3;
        node [shape = doublecircle] q3q4;
        node [shape = doublecircle] q3q5;
        node [shape = circle] q4q0;
        node [shape = circle] q4q1;
        node [shape = circle] q4q2;
        node [shape = circle] q4q3;
        node [shape = circle] q4q4;
        node [shape = circle] q4q5;
        start -> q0q0;
        
        q0q0 -> q1q3 [label = "a"] ;q0q0 -> q1q1 [label = "b"];
        q0q1 -> q1q0 [label = "a"] ;q0q1 -> q1q4 [label = "b"];
        q0q2 -> q1q5 [label = "a"] ;q0q2 -> q1q0 [label = "b"];
        q0q3 -> q1q2 [label = "a"] ;q0q3 -> q1q3 [label = "b"];
        q0q4 -> q1q1 [label = "a"] ;q0q4 -> q1q5 [label = "b"];
        q0q5 -> q1q4 [label = "a"] ;q0q5 -> q1q2 [label = "b"];
        q1q0 -> q1q3 [label = "a"] ;q1q0 -> q1q1 [label = "b"];
        q1q1 -> q1q0 [label = "a"] ;q1q1 -> q1q4 [label = "b"];
        q1q2 -> q1q5 [label = "a"] ;q1q2 -> q1q0 [label = "b"];
        q1q3 -> q1q2 [label = "a"] ;q1q3 -> q1q3 [label = "b"];
        q1q4 -> q1q1 [label = "a"] ;q1q4 -> q1q5 [label = "b"];
        q1q5 -> q1q4 [label = "a"] ;q1q5 -> q1q2 [label = "b"];
        q2q0 -> q1q3 [label = "a"] ;q2q0 -> q1q1 [label = "b"];
        q2q1 -> q1q0 [label = "a"] ;q2q1 -> q1q4 [label = "b"];
        q2q2 -> q1q5 [label = "a"] ;q2q2 -> q1q0 [label = "b"];
        q2q3 -> q1q2 [label = "a"] ;q2q3 -> q1q3 [label = "b"];
        q2q4 -> q1q1 [label = "a"] ;q2q4 -> q1q5 [label = "b"];
        q2q5 -> q1q4 [label = "a"] ;q2q5 -> q1q2 [label = "b"];
        q3q0 -> q1q3 [label = "a"] ;q3q0 -> q1q1 [label = "b"];
        q3q1 -> q1q0 [label = "a"] ;q3q1 -> q1q4 [label = "b"];
        q3q2 -> q1q5 [label = "a"] ;q3q2 -> q1q0 [label = "b"];
        q3q3 -> q1q2 [label = "a"] ;q3q3 -> q1q3 [label = "b"];
        q3q4 -> q1q1 [label = "a"] ;q3q4 -> q1q5 [label = "b"];
        q3q5 -> q1q4 [label = "a"] ;q3q5 -> q1q2 [label = "b"];
        q4q0 -> q1q3 [label = "a"] ;q4q0 -> q1q1 [label = "b"];
        q4q1 -> q1q0 [label = "a"] ;q4q1 -> q1q4 [label = "b"];
        q4q2 -> q1q5 [label = "a"] ;q4q2 -> q1q0 [label = "b"];
        q4q3 -> q1q2 [label = "a"] ;q4q3 -> q1q3 [label = "b"];
        q4q4 -> q1q1 [label = "a"] ;q4q4 -> q1q5 [label = "b"];
        q4q5 -> q1q4 [label = "a"] ;q4q5 -> q1q2 [label = "b"];
    }


\end{document}
