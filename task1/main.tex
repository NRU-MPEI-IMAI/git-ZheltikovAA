\documentclass[fleqn]{article}
\usepackage[utf8]{inputenc}
 \usepackage[russian]{babel}
 \usepackage[pdf]{graphviz}
 \usepackage{mathtools}
\title{Отчет по домашней работе номер 1}
\author{Александр Желтиков}
\usepackage[pdf]{graphviz}
\usepackage{xpatch}
\makeatletter
\newcommand*{\addFileDependency}[1]{% argument=file name and extension
  \typeout{(#1)}
  \@addtofilelist{#1}
  \IfFileExists{#1}{}{\typeout{No file #1.}}
}
\makeatother
\xpretocmd{\digraph}{\addFileDependency{#2.dot}}{}{}

\begin{document}
\chapter{\textbf{\large{Задание номер 1.}}}


\textbf{Построить конечный автомат, распознающий язык.}

Ответом на данное задание является конечный автомат, распознающий описанный язык. Автомат должен быть детерменированным.

\begin{equation}
1. \quad L = \{w \in \{ a,b,c \} ^*| \; |W|_c = 1 \}    \nonumber
\end{equation}
\digraph[scale=0.7]{1}{
        rankdir=LR;
        node [shape = none] start [label = < >];
        node [shape = circle] q0;
        node [shape = doublecircle] q1;
        start -> q0;
        q0->q1 [label = "c"] ;
        q0->q0 [label="a, b"] ; 
        q1->q1 [label = "a, b" ] ; 
    }
\begin{equation}
2. \quad L = \{w \in a,b^*| \; |W|_a \leq 2 , \; |W|_b \geq 2 \}    \nonumber
\end{equation}

\digraph[scale=0.7]{2}{
        rankdir=LR;
        node [shape = none] start [label = < >];
            node [shape = circle] q0;
            node [shape = circle] q1;
            node [shape = circle] q2;
            node [shape = circle] q3;
            node [shape = doublecircle] q4;
            node [shape = circle] q5;
            node [shape = doublecircle] q6;
            node [shape = circle] q7;
            node [shape = doublecircle] q8;
            start -> q0;
            q0 -> q1 [label = "b"] ; 
            q0 -> q2 [label = "a"] ;
            q1 -> q3 [label = "a"] ;
            q2 -> q3 [label = "b"] ; q4 -> q4 [label = "b"];
            q1 -> q4 [label = "b"] ; q6 -> q6 [label = "b"];
            q2 -> q5 [label = "a"] ; q8 -> q8 [label = "b"];
            q4 -> q6 [label = "a"] ;
            q3 -> q6 [label = "b"] ;
            q5 -> q7 [label = "b"] ;
            q7 -> q8 [label = "b"] ;
            q6 -> q8 [label = "a"] ;
        } 

\begin{equation}
3. \quad L = \{w \in \{ a,b\} ^*| \; |W|_a \neq |W|_b \}    \nonumber
\end{equation}
Язык не является регулярным => по нему нельзя построить детерминированный конечный автомат.
\\

З.Ы. Потому что автоматы бесславные ублюдки (глупцы).
\begin{equation}
4. \quad L = \{w \in a,b ^*| \; ww = www\}    \nonumber
\end{equation}

\digraph[scale=0.7]{4}{
        rankdir=LR;
        node [shape = none] start [label = < >];
        node [shape = doublecircle] q0;
        start -> q0;
    }

\end{document}
